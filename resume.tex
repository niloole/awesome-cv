%%%%%%%%%%%%%%%%%%%%%%%%%%%%%%%%%%%%%%%%%
% ‎Medium Length Graduate Curriculum Vitae‎
% ‎LaTeX Template‎
% ‎Version 1.1 (9/12/12)‎
%
% ‎This template has been downloaded from‎:
% ‎http://www.LaTeXTemplates.com‎
%
% ‎Original author‎:
% ‎Rensselaer Polytechnic Institute (http://www.rpi.edu/dept/arc/training/latex/resumes/)‎
%
% ‎Important note‎:
% ‎This template requires the res.cls file to be in the same directory as the‎
% .‎tex file‎. ‎The res.cls file provides the resume style used for structuring the‎
% ‎document‎.
%
%%%%%%%%%%%%%%%%%%%%%%%%%%%%%%%%%%%%%%%%%

%----------------------------------------------------------------------------------------
‎%	PACKAGES AND OTHER DOCUMENT CONFIGURATIONS‎
%----------------------------------------------------------------------------------------

‎\documentclass[margin‎, ‎10pt]{res} % Use the res.cls style‎, ‎the font size can be changed to 11pt or 12pt here‎
‎\usepackage{hyperref}‎
‎\usepackage{helvet} % Default font is the helvetica postscript font‎
‎%\usepackage{newcent} % To change the default font to the new century schoolbook postscript font uncomment this line and comment the one above‎

‎\setlength{\textwidth}{5.1in} % Text width of the document‎

‎\begin{document}‎
‎\moveleft.5\hoffset\centerline{\large\bf Niloofar Borhani} % Your name at the top‎
‎\moveleft\hoffset\vbox{\hrule width\resumewidth height 1pt}\smallskip % Horizontal line after name; adjust line thickness by changing the '1pt'‎
‎\moveleft.5\hoffset\centerline{borhaniniloofar1374@gmail.com} % Your address‎
‎\moveleft.5\hoffset\centerline{‎
‎\href{https://niloole.github.io/awesome-cv/}{(website CV‎: https://niloole.github.io/awesome-cv/)}‎
}‎
‎\moveleft.5\hoffset\centerline{+98 913 320 3525}‎

‎\begin{resume}‎
‎%\section{OBJECTIVE}‎  
‎%A position in the field of computers with special interests in business applications programming‎, ‎information processing‎, ‎and management systems‎. 
‎\section{EDUCATION}‎

‎{\sl \textbf{Master of Science,}} \textbf{Isfahan University of Technology }\\‎
‎\textbf{Control Engineering, ‎2017‎ - ‎2020} \\‎
‎\textbf{GPA‎: ‎18.64/20 (4/4)} \\‎
‎\textbf{Thesis title:} Data integration for prediction of inter-omics layers interactions in multi-layer networks using matrix factorization\\‎
‎\textbf{Supervisors:} Dr.Jafar Ghaisari and Dr.Marzieh Kamali\\‎
‎\textbf{Advisor:} Dr.Yousof Gheisar\\‎

‎{\sl \textbf{Bachelor of Science,}}\textbf{ Isfahan University of Technology}\\‎
‎\textbf{Control Engineering,‎ ‎2013 – 2017}\\‎
‎\textbf{GPA‎: ‎18.45/20 (4/4)}\\‎
‎\textbf{Thesis title:} Farsi handwritten recognition using ensemble of neural networks\\‎
‎\textbf{Supervisor:} Dr.Marzieh Kamali‎
 
‎\section{HONOURS \& AWARDS}‎
‎\begin{itemize}‎ 
‎\item Ranking 2 among control engineering students in the department of electrical engineering at Isfahan University of Technology‎ 
‎\item Among the top $ 10 \% $ of undergraduate electrical engineering (ranking 2 among undergraduate control engineering) in the department of electrical engineering at Isfahan University of Technology in 2017‎ 
‎\item Among the best students of Isfahan province in the diploma
‎\item Among top $ 1 \% $ of participates in Iranian University entrance exam for master degree‎
‎\item Received national graduate and undergraduate full scholarship‎
‎\end{itemize}‎ 

‎\section{RESEARCH INTERESTS}‎
‎\begin{itemize}‎ 
‎\item Machine learning‎, ‎Deep learning‎
‎\item Node embedding‎
‎\item Representation learning‎
‎\item Data Science‎
‎\item Systems biology‎
‎\item Systems modeling‎ 
‎\end{itemize}‎ 

‎\section{ACADEMIC PROJECTS}‎
‎\begin{itemize}‎ 
‎\item miRNA targets prediction with computational methods‎
‎\item Data integration for prediction of inter-omics layers interactions in multi-layer networks using matrix factorization‎
‎\item Prediction of gene ontology by deep learning‎ 
‎\item Modeling corona virus transmission with the agent based method‎
‎\item Introduction to modeling biological systems (ODE‎, ‎Petri nets‎, etc‎)‎
‎\item Campbell's Biology Summary in Simple Language‎
‎\item Implementation adaptive controllers for controlling speed‎
‎\item Farsi handwritten recognition using ensemble of neural networks
‎\item Using Bluetooth module and proximity sensors for RGB LED‎ 
‎\item Construction of Buck converter‎
‎\item Construction of FM receiver‎
  
‎\end{itemize}‎ 
 
 
‎\section{PUBLICATION}‎
‎A deep learning method for miRNA targets prediction (In prep)‎

 
‎\section{EXPERIENCE}‎
‎{\sl \textbf{Teacher assessment}} \\‎
 ‎Isfahan University of Technology‎, ‎In digital control laboratory\\‎
 ‎January 2018 – June 2018 \\‎
 ‎January 2019 – June 2019‎ 
 
‎{\sl \textbf{Reader assessment}} \\‎
 ‎Isfahan University of Technology \\‎
‎Research methodology‎, ‎June 2018 \\‎
E‎lectrical circuit \rom{2}‎, ‎January 2019‎ 
%
‎%\begin{itemize} \itemsep‎ -‎2pt % Reduce space between items‎
‎%\item Developed four ``user friendly‎" ‎forecasting systems each of which produces 18 to 139 individual reports‎. 
‎%\item Developed or improved almost all IFPS programs used for financial reports‎. 
‎%\end{itemize}‎
% 
‎\section{COMPUTER \\ SKILLS}‎ 

‎{\sl \textbf{Programming:}}‎ 
‎Python‎, ‎MATLAB‎, ‎C‎, ‎VERILOG‎
‎\vspace{0.2cm}\\‎
‎{\sl \textbf{Software:}}‎
‎Microsoft Office‎, ‎\LaTeX‎ , ‎AutoCAD‎, LabVIEW, ‎AVR and Proteus‎, ‎SIMATIC Manager and WinCC‎ 

‎\section{LANGUAGE}‎
‎{\sl \textbf{Persian:}} Native\\‎
‎{\sl \textbf{English:}} Fluent (TOEFL iBT‎: ‎will be taken on Sep 2020)‎
             
             
‎\section{MEMBERSHIP}‎ 
‎Regenerative Medicine Research center at Isfahan University of Medical Sciences‎
‎\href{https://rml.mui.ac.ir/}{(website‎: ‎https://rml.mui.ac.ir/)}‎

‎\section{REFERENCES}‎
‎\textbf{Dr‎. ‎Jafar Ghaisari‎ }
‎Associative Professor‎, ‎Department of Electrical and Computer Engineering‎, ‎Isfahan University of Technology‎, ‎Isfahan‎, ‎Iran‎, ‎\underline{ghaisari@cc.iut.ac.ir}‎ 

‎\textbf{Dr‎. ‎Marzieh Kamali‎ } 
‎Assistant Professor‎, ‎Department of Electrical and Computer Engineering‎, ‎Isfahan University of Technology‎, ‎Isfahan‎, ‎Iran‎, ‎\underline{m.kamali@cc.iut.ac.ir}‎ 

‎\textbf{Dr‎. ‎Yousof Gheisar‎ }
‎MD‎, ‎PhD‎, ‎Associative Professor‎, ‎Isfahan University of Medical Sciences‎, ‎Isfahan‎, ‎Iran‎, ‎\underline{ygheisari@med.mui.ac.ir}‎
‎\end{resume}‎
‎\end{document}
